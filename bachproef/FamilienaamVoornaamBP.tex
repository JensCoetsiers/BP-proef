%===============================================================================
% LaTeX sjabloon voor de bachelorproef toegepaste informatica aan HOGENT
% Meer info op https://github.com/HoGentTIN/latex-hogent-report
%===============================================================================

\documentclass[dutch,dit,thesis]{hogentreport}

% TODO:
% - If necessary, replace the option `dit`' with your own department!
%   Valid entries are dbo, dbt, dgz, dit, dlo, dog, dsa, soa
% - If you write your thesis in English (remark: only possible after getting
%   explicit approval!), remove the option "dutch," or replace with "english".

\usepackage{lipsum} % For blind text, can be removed after adding actual content

%% Pictures to include in the text can be put in the graphics/ folder
\graphicspath{{graphics/}}

%% For source code highlighting, requires pygments to be installed
%% Compile with the -shell-escape flag!
\usepackage[section]{minted}
%% If you compile with the make_thesis.{bat,sh} script, use the following
%% import instead:
%% \usepackage[section,outputdir=../output]{minted}
\usemintedstyle{solarized-light}
\definecolor{bg}{RGB}{253,246,227} %% Set the background color of the codeframe

%% Change this line to edit the line numbering style:
\renewcommand{\theFancyVerbLine}{\ttfamily\scriptsize\arabic{FancyVerbLine}}

%% Macro definition to load external java source files with \javacode{filename}:
\newmintedfile[javacode]{java}{
    bgcolor=bg,
    fontfamily=tt,
    linenos=true,
    numberblanklines=true,
    numbersep=5pt,
    gobble=0,
    framesep=2mm,
    funcnamehighlighting=true,
    tabsize=4,
    obeytabs=false,
    breaklines=true,
    mathescape=false
    samepage=false,
    showspaces=false,
    showtabs =false,
    texcl=false,
}

% Other packages not already included can be imported here
\usepackage{epigraph}

%%---------- Document metadata -------------------------------------------------
% TODO: Replace this with your own information
\author{Jens Coetsiers}
\supervisor{Dhr. S. Lievens}
\cosupervisor{Mevr. L. De Mol}
\title
    {Optimaliseren van transcriptie voor (niet-Standaard) Nederlands in telefoongesprekken, een onderzoek naar speech-to-text systemen}
\academicyear{\advance\year by -1 \the\year--\advance\year by 1 \the\year}
\examperiod{1}
\degreesought{\IfLanguageName{dutch}{Professionele bachelor in de toegepaste informatica}{Bachelor of applied computer science}}
\partialthesis{false} %% To display 'in partial fulfilment'
%\institution{Internshipcompany BVBA.}

%% Add global exceptions to the hyphenation here
\hyphenation{back-slash}

%% The bibliography (style and settings are  found in hogentthesis.cls)
\addbibresource{bachproef.bib}            %% Bibliography file
\addbibresource{../voorstel/bib.bib} %% Bibliography research proposal
\defbibheading{bibempty}{}

%% Prevent empty pages for right-handed chapter starts in twoside mode
\renewcommand{\cleardoublepage}{\clearpage}

\renewcommand{\arraystretch}{1.2}

%% Content starts here.
\begin{document}

%---------- Front matter -------------------------------------------------------

\frontmatter

\hypersetup{pageanchor=false} %% Disable page numbering references
%% Render a Dutch outer title page if the main language is English
\IfLanguageName{english}{%
    %% If necessary, information can be changed here
    \degreesought{Professionele Bachelor toegepaste informatica}%
    \begin{otherlanguage}{dutch}%
       \maketitle%
    \end{otherlanguage}%
}{}

%% Generates title page content
\maketitle
\hypersetup{pageanchor=true}

%%=============================================================================
%% Voorwoord
%%=============================================================================

\chapter*{\IfLanguageName{dutch}{Woord vooraf}{Preface}}%
\label{ch:voorwoord}

%% TODO:
%% Het voorwoord is het enige deel van de bachelorproef waar je vanuit je
%% eigen standpunt (``ik-vorm'') mag schrijven. Je kan hier bv. motiveren
%% waarom jij het onderwerp wil bespreken.
%% Vergeet ook niet te bedanken wie je geholpen/gesteund/... heeft

\lipsum[1-2]
%%=============================================================================
%% Samenvatting
%%=============================================================================

% TODO: De "abstract" of samenvatting is een kernachtige (~ 1 blz. voor een
% thesis) synthese van het document.
%
% Een goede abstract biedt een kernachtig antwoord op volgende vragen:
%
% 1. Waarover gaat de bachelorproef?
% 2. Waarom heb je er over geschreven?
% 3. Hoe heb je het onderzoek uitgevoerd?
% 4. Wat waren de resultaten? Wat blijkt uit je onderzoek?
% 5. Wat betekenen je resultaten? Wat is de relevantie voor het werkveld?
%
% Daarom bestaat een abstract uit volgende componenten:
%
% - inleiding + kaderen thema
% - probleemstelling
% - (centrale) onderzoeksvraag
% - onderzoeksdoelstelling
% - methodologie
% - resultaten (beperk tot de belangrijkste, relevant voor de onderzoeksvraag)
% - conclusies, aanbevelingen, beperkingen
%
% LET OP! Een samenvatting is GEEN voorwoord!

%%---------- Nederlandse samenvatting -----------------------------------------
%
% TODO: Als je je bachelorproef in het Engels schrijft, moet je eerst een
% Nederlandse samenvatting invoegen. Haal daarvoor onderstaande code uit
% commentaar.
% Wie zijn bachelorproef in het Nederlands schrijft, kan dit negeren, de inhoud
% wordt niet in het document ingevoegd.

\IfLanguageName{english}{%
\selectlanguage{dutch}
\chapter*{Samenvatting}
\lipsum[1-4]
\selectlanguage{english}
}{}

%%---------- Samenvatting -----------------------------------------------------
% De samenvatting in de hoofdtaal van het document

\chapter*{\IfLanguageName{dutch}{Samenvatting}{Abstract}}

\lipsum[1-4]


%---------- Inhoud, lijst figuren, ... -----------------------------------------

\tableofcontents

% In a list of figures, the complete caption will be included. To prevent this,
% ALWAYS add a short description in the caption!
%
%  \caption[short description]{elaborate description}
%
% If you do, only the short description will be used in the list of figures

\listoffigures

% If you included tables and/or source code listings, uncomment the appropriate
% lines.
%\listoftables
%\listoflistings

% Als je een lijst van afkortingen of termen wil toevoegen, dan hoort die
% hier thuis. Gebruik bijvoorbeeld de ``glossaries'' package.
% https://www.overleaf.com/learn/latex/Glossaries

%---------- Kern ---------------------------------------------------------------

\mainmatter{}

% De eerste hoofdstukken van een bachelorproef zijn meestal een inleiding op
% het onderwerp, literatuurstudie en verantwoording methodologie.
% Aarzel niet om een meer beschrijvende titel aan deze hoofdstukken te geven of
% om bijvoorbeeld de inleiding en/of stand van zaken over meerdere hoofdstukken
% te verspreiden!

%%=============================================================================
%% Inleiding
%%=============================================================================

\chapter{\IfLanguageName{dutch}{Inleiding}{Introduction}}%
\label{ch:inleiding}

Spraak-naar-tekst technologie, ook bekend als Spraakherkenning (SR), is een integraal onderdeel geworden van ons dagelijks leven en biedt de mogelijkheid om gesproken taal om te zetten in tekst. Met de vooruitgang in kunstmatige intelligentie (AI), met name op het gebied van deep learning, hebben SR-systemen een opmerkelijke nauwkeurigheid en functionaliteit bereikt, waardoor een breed scala aan toepassingen mogelijk is geworden, zoals automatische ondertiteling en parsing van spraakopdrachten \autocite{Roepke2019}.

\section{Context en achtergrond}
Dit onderzoek richt zich op Speech-to-Text (STT) technologie om de praktische toepassingen en mogelijke maatschappelijke gevolgen te onderzoeken. Het belang en de relevantie van STT in de huidige samenleving worden onderstreept door zijn uitgebreide toepassingen, die variëren van gezondheidszorghulp tot kinderen helpen met problemen in gesproken en geschreven communicatie \autocite{Kambouri2023}. Zoals \textcite{Shakhovska2019} ook aanduidt zullen nieuwe spraakherkenningstechnologieën en -methodes een centraal onderdeel
worden van het toekomstige leven omdat ze veel communicatietijd besparen en het gewone sms’en vervangen door spraak/audio.


\section{Probleemstelling}
Hoewel er aanzienlijke vooruitgang is geboekt in spraakherkenningstechnologie, blijven bepaalde problemen bestaan. De beschikbaarheid van commerciële spraakherkenningssystemen in het Nederlands is volgens het onderzoek van \textcite{Wei2022} beperkt alsook de uitgebreide variatie en diversiteit aan regionale accenten zoals ook aangekaart in het onderzoek van \autocite{ghyselen2020clearing} en \autocite{barbiers2004reflexieven} vormt mogelijks een uitdaging. Dit leidt tot de vraag of huidige spraak-naar-tekst (STT) technologieën het Nederlands, inclusief regionale accenten en verschillen in taalgebruik, kunnen beheersen. Als gevolg hiervan zal het onderzoek de aanpassingsmogelijkheden en nauwkeurigheid van bestaande STT-systemen onderzoeken bij het vastleggen van verschillende linguïstische nuances in het Nederlands. Het doel is om inzicht te krijgen in de prestaties van STT-systemen in het Nederlands en potentiële verbeterpunten te vinden om de technologie in het Nederlands beter te gebruiken.

\section{Onderzoeksdoelstellingen}

De primaire onderzoeksdoelstellingen omvatten:

\begin{itemize}
    \item Het onderzoeken van de invloed van omgevingsgeluid op STT-nauwkeurigheid.
    \item Beoordelen van de geschiktheid van modellen getraind op verschillende talen voor STT-taken in het Nederlands.
    \item Het identificeren van open source en closed source modellen voor Nederlandse taalverwerking.
    \item Hoe presteert een geoptimaliseerd model op woorden die niet behoren tot de standaardtaal adhv. word error rate?
    \item Het evalueren van de haalbaarheid en efficiëntie van het implementeren van verbeterde STT-modellen in de praktijk.
\end{itemize}

\section{Onderzoeksvraag}
Wees zo concreet mogelijk bij het formuleren van je onderzoeksvraag. Een onderzoeksvraag is trouwens iets waar nog niemand op dit moment een antwoord heeft (voor zover je kan nagaan). Het opzoeken van bestaande informatie (bv. ``welke tools bestaan er voor deze toepassing?'') is dus geen onderzoeksvraag. Je kan de onderzoeksvraag verder specifiëren in deelvragen. Bv.~als je onderzoek gaat over performantiemetingen, dan ...

\section{Methodologie van het onderzoek}
Om deze doelstellingen te bereiken, zal het onderzoek verschillende fasen doorlopen. Het zal beginnen met een uitgebreid literatuuronderzoek om het huidige landschap van STT-technologie in kaart te brengen en bestaande uitdagingen te identificeren. Vervolgens wordt in het hoofdstuk over methodologie de aanpak beschreven voor het onderzoeken en verbeteren van de transcriptiekwaliteit, met een specifieke focus op gesprekken met oudere mensen en studenten van HOGENT. In dit gedeelte worden het onderzoeksdesign, de methoden voor gegevensverzameling en de criteria voor de evaluatie van het model toegelicht.

Betekenis en verwachte resultaten
De resultaten van dit onderzoek zijn bedoeld om bij te dragen aan de vooruitgang van STT-technologie, in het bijzonder bij het verwerken van regionale accenten en (niet-standaard) Nederlands. De ontwikkeling van een geoptimaliseerd STT-model op maat voor Nederlands en taalvariaties wordt verwacht. Verder zullen de inzichten uit het onderzoek licht werpen op de uitdagingen en beperkingen in de huidige SR-systemen, waardoor de weg wordt vrijgemaakt voor toekomstig onderzoek en technologische verbeteringen in dit domein.

Het doel van dit onderzoek is om de mogelijkheden en beperkingen van STT-technologie te onderzoeken bij het aanpakken van de taalkundige diversiteit in het Nederlands. Het laatste doel is om de bruikbaarheid en nauwkeurigheid van transcriberen in het Nederlands te vergroten.





\section{\IfLanguageName{dutch}{Probleemstelling}{Problem Statement}}%
\label{sec:probleemstelling}

Uit je probleemstelling moet duidelijk zijn dat je onderzoek een meerwaarde heeft voor een concrete doelgroep. De doelgroep moet goed gedefinieerd en afgelijnd zijn. Doelgroepen als ``bedrijven,'' ``KMO's'', systeembeheerders, enz.~zijn nog te vaag. Als je een lijstje kan maken van de personen/organisaties die een meerwaarde zullen vinden in deze bachelorproef (dit is eigenlijk je steekproefkader), dan is dat een indicatie dat de doelgroep goed gedefinieerd is. Dit kan een enkel bedrijf zijn of zelfs één persoon (je co-promotor/opdrachtgever).



\section{\IfLanguageName{dutch}{Onderzoeksdoelstelling}{Research objective}}%
\label{sec:onderzoeksdoelstelling}
Wat is het beoogde resultaat van je bachelorproef? Wat zijn de criteria voor succes? Beschrijf die zo concreet mogelijk. Gaat het bv.\ om een proof-of-concept, een prototype, een verslag met aanbevelingen, een vergelijkende studie, enz.

\section{\IfLanguageName{dutch}{Opzet van deze bachelorproef}{Structure of this bachelor thesis}}%
\label{sec:opzet-bachelorproef}

% Het is gebruikelijk aan het einde van de inleiding een overzicht te
% geven van de opbouw van de rest van de tekst. Deze sectie bevat al een aanzet
% die je kan aanvullen/aanpassen in functie van je eigen tekst.

De rest van deze bachelorproef is als volgt opgebouwd:

In Hoofdstuk~\ref{ch:stand-van-zaken} wordt een overzicht gegeven van de stand van zaken binnen het onderzoeksdomein, op basis van een literatuurstudie.

In Hoofdstuk~\ref{ch:methodologie} wordt de methodologie toegelicht en worden de gebruikte onderzoekstechnieken besproken om een antwoord te kunnen formuleren op de onderzoeksvragen.

% TODO: Vul hier aan voor je eigen hoofstukken, één of twee zinnen per hoofdstuk

In Hoofdstuk~\ref{ch:conclusie}, tenslotte, wordt de conclusie gegeven en een antwoord geformuleerd op de onderzoeksvragen. Daarbij wordt ook een aanzet gegeven voor toekomstig onderzoek binnen dit domein.
\chapter{\IfLanguageName{dutch}{Stand van zaken}{State of the art}}%
\label{ch:stand-van-zaken}

% Tip: Begin elk hoofdstuk met een paragraaf inleiding die beschrijft hoe
% dit hoofdstuk past binnen het geheel van de bachelorproef. Geef in het
% bijzonder aan wat de link is met het vorige en volgende hoofdstuk.

% Pas na deze inleidende paragraaf komt de eerste sectiehoofding.

Dit hoofdstuk bevat je literatuurstudie. De inhoud gaat verder op de inleiding, maar zal het onderwerp van de bachelorproef *diepgaand* uitspitten. De bedoeling is dat de lezer na lezing van dit hoofdstuk helemaal op de hoogte is van de huidige stand van zaken (state-of-the-art) in het onderzoeksdomein. Iemand die niet vertrouwd is met het onderwerp, weet nu voldoende om de rest van het verhaal te kunnen volgen, zonder dat die er nog andere informatie moet over opzoeken \autocite{Pollefliet2011}.

Je verwijst bij elke bewering die je doet, vakterm die je introduceert, enz.\ naar je bronnen. In \LaTeX{} kan dat met het commando \texttt{$\backslash${textcite\{\}}} of \texttt{$\backslash${autocite\{\}}}. Als argument van het commando geef je de ``sleutel'' van een ``record'' in een bibliografische databank in het Bib\LaTeX{}-formaat (een tekstbestand). Als je expliciet naar de auteur verwijst in de zin (narratieve referentie), gebruik je \texttt{$\backslash${}textcite\{\}}. Soms is de auteursnaam niet expliciet een onderdeel van de zin, dan gebruik je \texttt{$\backslash${}autocite\{\}} (referentie tussen haakjes). Dit gebruik je bv.~bij een citaat, of om in het bijschrift van een overgenomen afbeelding, broncode, tabel, enz. te verwijzen naar de bron. In de volgende paragraaf een voorbeeld van elk.

\textcite{Knuth1998} schreef een van de standaardwerken over sorteer- en zoekalgoritmen. Experten zijn het erover eens dat cloud computing een interessante opportuniteit vormen, zowel voor gebruikers als voor dienstverleners op vlak van informatietechnologie~\autocite{Creeger2009}.

Let er ook op: het \texttt{cite}-commando voor de punt, dus binnen de zin. Je verwijst meteen naar een bron in de eerste zin die erop gebaseerd is, dus niet pas op het einde van een paragraaf.

\lipsum[7-20]

%%=============================================================================
%% Methodologie
%%=============================================================================

\chapter{\IfLanguageName{dutch}{Methodologie}{Methodology}}%
\label{ch:methodologie}

%% TODO: In dit hoofstuk geef je een korte toelichting over hoe je te werk bent
%% gegaan. Verdeel je onderzoek in grote fasen, en licht in elke fase toe wat
%% de doelstelling was, welke deliverables daar uit gekomen zijn, en welke
%% onderzoeksmethoden je daarbij toegepast hebt. Verantwoord waarom je
%% op deze manier te werk gegaan bent.
%% 
%% Voorbeelden van zulke fasen zijn: literatuurstudie, opstellen van een
%% requirements-analyse, opstellen long-list (bij vergelijkende studie),
%% selectie van geschikte tools (bij vergelijkende studie, "short-list"),
%% opzetten testopstelling/PoC, uitvoeren testen en verzamelen
%% van resultaten, analyse van resultaten, ...
%%
%% !!!!! LET OP !!!!!
%%
%% Het is uitdrukkelijk NIET de bedoeling dat je het grootste deel van de corpus
%% van je bachelorproef in dit hoofstuk verwerkt! Dit hoofdstuk is eerder een
%% kort overzicht van je plan van aanpak.
%%
%% Maak voor elke fase (behalve het literatuuronderzoek) een NIEUW HOOFDSTUK aan
%% en geef het een gepaste titel.

\lipsum[21-25]



% Voeg hier je eigen hoofdstukken toe die de ``corpus'' van je bachelorproef
% vormen. De structuur en titels hangen af van je eigen onderzoek. Je kan bv.
% elke fase in je onderzoek in een apart hoofdstuk bespreken.

%\input{...}
%\input{...}
%...

%%=============================================================================
%% Conclusie
%%=============================================================================

\chapter{Conclusie}%
\label{ch:conclusie}

% TODO: Trek een duidelijke conclusie, in de vorm van een antwoord op de
% onderzoeksvra(a)g(en). Wat was jouw bijdrage aan het onderzoeksdomein en
% hoe biedt dit meerwaarde aan het vakgebied/doelgroep? 
% Reflecteer kritisch over het resultaat. In Engelse teksten wordt deze sectie
% ``Discussion'' genoemd. Had je deze uitkomst verwacht? Zijn er zaken die nog
% niet duidelijk zijn?
% Heeft het onderzoek geleid tot nieuwe vragen die uitnodigen tot verder 
%onderzoek?

\lipsum[76-80]



%---------- Bijlagen -----------------------------------------------------------

\appendix

\chapter{Onderzoeksvoorstel}

Het onderwerp van deze bachelorproef is gebaseerd op een onderzoeksvoorstel dat vooraf werd beoordeeld door de promotor. Dat voorstel is opgenomen in deze bijlage.
\vspace{3pt}

%% TODO: 
%\section*{Samenvatting}

% Kopieer en plak hier de samenvatting (abstract) van je onderzoeksvoorstel.
Speech-to-text, ook wel speech recognition genoemd, is een technologie die luistert naar verbale geluidsopnamen en deze omzet in een geschreven script, dat vervolgens kan worden gebruikt in verschillende toepassingen. Dit onderzoek heeft als doel de transcriptiekwaliteit te verbeteren van gesprekken waarin (geen standaard) Nederlands wordt gebruikt, inclusief regionale accenten. De gesprekken vinden plaats tussen oudere personen die worden gebeld door studenten van HOGENT. Het model zou worden gebruikt om een transcriptie te leveren van deze gesprekken, waarbij er een leeftijdsverschil is tussen de bellers en de deelnemers aan het onderzoek.

Eerst en vooral zullen de prestaties van zowel commerciële als open source speech-to-text systemen worden onderzocht, waarbij we de invloed van finetuning op deze systemen evalueren met als doel de transcriptiekwaliteit te verbeteren. Hierbij staat de volgende vraag centraal: "Kunnen we de speech-to-text technologie verbeteren om beter om te gaan met Nederlands, inclusief regionale accenten?". Dit onderzoek beoogt inzicht te verschaffen in de kwaliteit van transcripties in (niet-standaard) Nederlands, waarbij slechte kwaliteit als obstakel kan dienen voor diverse toepassingen. Hiermee dragen we bij aan de ontwikkeling van verbeterde transcriptiediensten voor (niet-standaard) Nederlands, bevorderen we taalbehoud en ondersteunen we taalkundig onderzoek, met de focus op het verbeteren van de kwaliteit van transcripties in spraakopdrachten.

De methodologie omvat prestatie-evaluaties van zowel commerciële als open source speech-to-text systemen, evenals het finetunen van deze modellen indien mogelijk. Volgens verschillende studies kunnen technieken zoals transfer learning en optimalisatie van hyperparameters de transcriptiekwaliteit verbeteren.

In conclusie biedt dit onderzoek nieuwe inzichten en praktische oplossingen voor het verbeteren van de transcriptiekwaliteit van dergelijke systemen. De bevindingen dragen bij aan de verbetering van spraakherkenningstechnologieën, met positieve implicaties voor toepassingsgebieden zoals klantenservice en interculturele communicatie.

% Verwijzing naar het bestand met de inhoud van het onderzoeksvoorstel
%---------- Inleiding ---------------------------------------------------------
\section{Introductie}%
\label{sec:introductie}
\epigraph{Nieuwe spraakherkenningstechnologieën en -methodes zullen een centraal onderdeel worden van het toekomstige leven omdat ze veel communicatietijd besparen en het gewone sms'en vervangen door spraak/audio.}{Natalya Shakhovska}

In het bovenstaande voorbeeld wordt duidelijk geïllustreerd hoe belangrijk speech-to-text is in ons dagelijks leven, zowel in het nu als in de toekomst. Deze technologie speelt een steeds grotere rol in onze snel evoluerende wereld doordat het ons in staat stelt gesproken taal nauwkeurig om te zetten in geschreven tekst. De invloed van speech-to-text is tegenwoordig al in gebruik met een breed scala aan toepassingen zoals:
\begin{itemize}
    \item kinderen helpen met problemen in gesproken, geschreven communicatie \cite{Kambouri2023}.
    \item automatische ondertiteling
    \item Google Assistant, Siri.
\end{itemize}

In dit onderzoek ligt de focus voornamelijk op spraakopdrachten, zoals die in gesprekken tussen klant en bedrijf of tussen vrienden voorkomen. Hoewel dergelijke systemen al aanzienlijke vooruitgang hebben geboekt, blijft de verwerking van dialecten en regionale accenten een uitdaging voor deze technologieën.

\subsection{Uitdagingen bij Niet-standaard Nederlands}
Het Nederlands taalgebied heeft een rijke variatie aan dialecten. Dat vormt dus een uitdaging voor speech to text systemen die oorspronkelijk getraind zijn op standaard Nederlands. De vraag die dan gesteld kan worden is: Is deze technologie instaat om deze variatie en diversiteit te begrijpen, en of ze kunnen worden verbeterd om nauwkeurige transcripties te leveren voor niet-standaard Nederlands?
Deze vraag zal de kern vormen van ons onderzoek en zal dienen als leidraad

Vanuit bovenstaand idee kwam het doel om een effectieve oplossing te bieden voor de problematische transcriptie van spraak in niet-standaard Nederlands, waarbij de focus ligt op dialecten, regionale accenten en het identificeren van de spreker, ook wanneer het resultaat niet is zo als we verwacht hadden kunnen we toch inzichten verwerven en een nuttige bijdrage leveren.

Het concrete resultaat zal bestaan uit een (aangepast) model voor spraakherkenning dat geoptimaliseerd is voor niet-standaard Nederlands. Echter, in het geval dat het ontwikkelen of aanpassen van een dergelijk model niet succesvol blijkt te zijn, zal het onderzoeksrapport een gedetailleerde evaluatie bevatten van de uitdagingen en beperkingen die zijn geïdentificeerd.

\subsection{Opzet van het onderzoek}
Dit onderzoek zal zich ontvouwen in verschillende onderdelen, beginnend met een grondige literatuurstudie om de huidige stand van zaken in het vakgebied te begrijpen. Vervolgens zal de methodologie worden beschreven, waarbij we onze aanpak voor het verkennen en verbeteren van de transcriptiekwaliteit zullen uitschrijven. Ten slotte zullen we de verwachte resultaten bespreken.



%---------- Stand van zaken ---------------------------------------------------

\section{State-of-the-art}%
\label{sec:state-of-the-art}

Speech to text technologie ook gekend als speech recognition, heeft al een opvallende vooruitgang geboekt over verschillende domeinen heen. Dit kunnen we bevestigen uit het onderzoek van \cite{Kambouri2023} die aantoont hoe kinderen met schrijfproblemen voordeel kunnen halen uit deze technologie. Het verbeterde de kwaliteit en hoeveelheid van hun geschreven en gesproken communicatie.

Niettemin zien we uit het onderzoek van \cite{ajami2016use} dat deze technologie zeer succesvol is in de medische sector, waar het veel tijd en geld kan besparen. Deze sector heeft vaak te kampen met onjuiste registratie van gegevens, onleesbaarheid en ongeldige gegevens. Deze beperkingen brengen de kwaliteit van medische dossiers in gevaar en kunnen ernstige gevolgen hebben. Het onderzoek benadrukt dat spraakherkenningstechnologie de tijd die nodig is om medische dossiers in te vullen aanzienlijk kan verkorten, wat leidt tot een hogere productiviteit en lagere documentatiekosten.

\subsection{Uitdagingen}
Een van de belangrijkste uitdagingen die uit onderzoeken naar voren komen is de moeilijkheid om regionale accenten en gemengde talen te herkennen. 
Zoals beschreven door \cite{Reddy2022}, vormt meertalige speech to text technologie een unieke uitdaging, vooral als mensen zinnen gebruiken die een mix van talen bevatten. 
Dit vormt een obstakel voor effectieve communicatie tussen sprekers van verschillende talen of dialecten. Daarnaast wordt ook in het onderzoek van \cite{ajami2016use} benadrukt dat omgevingsgeluid problemen kan veroorzaken met de nauwkeurigheid van spraakherkenning, waardoor verdere verbeteringen op dit gebied nodig zijn.
Deze uitdagingen geven aanleiding tot belangrijke vragen voor toekomstig onderzoek: Kunnen we speech to text technologie verbeteren om beter om te gaan met regionale accenten en dialecten? Is er een manier om omgevingsgeluid te verminderen om de nauwkeurigheid van de technologie te vergroten?


\subsection{Vooruitgang}
Spraak is 1 van de meest gebruikte vormen van communicatie en heeft daarom ook een groot aantal toepassingen, zeker voor analfabeten en zorginstellingen \cite{Arun2021}. Volgens het onderzoek van \cite{Roepke2019} kan het gebruik van speech to text technologie een grote invloed hebben op de maatschappij. Datzelfde onderzoek heeft ook het belang van transfer learning aangeduid voor kleinere datasets te gebruiken en toch een succesvol resultaat te behalen. Uit zowat alle onderzoeken wordt de behoefte aan dergelijke systemen benadrukt, en onderstreept de rol van spraak- en taalverwerking om computers instaat te stellen menselijke talen te begrijpen en te interpreteren.

\subsection{Conclusie}
Speech to text technologie heeft het potentieel om vooruitgang te boeken in diverse sectoren, waaronder gezondheidszorg, onderwijs en interculturele communicatie. Zoals deze literatuurstudie heeft aangetoond, zijn er toch nog aanzienlijke uitdagingen die moeten worden overwonnen. Toekomstig onderzoek kan gericht zijn op het verbeteren van de nauwkeurigheid van de technologie, het overbruggen van taalbarrières en het verbeteren van communicatie voor mensen met speciale behoeften.

Deze literatuurstudie sluit nauw aan bij het gekozen onderzoek, dat zich richt op het optimaliseren van transcriptie voor niet-standaard Nederlands in gesprekken. Daarnaast biedt het de mogelijkheid om te onderzoeken hoe omgevingsgeluid effectief kan worden gedempt om de nauwkeurigheid van speech to text technologie te vergroten.


%---------- Methodologie ------------------------------------------------------
\section{Methodologie}%
\label{sec:methodologie}

We zullen het onderzoek uitvoeren in zes stappen, waarbij de eerste fase gericht is op het vinden van relevante bronnen voor het onderzoek. De volgende stappen worden dan gevolgd.

\begin{itemize}
    \item Long list: De long list zal bestaan uit alle bronnen die mogelijk relevant zijn voor het onderzoek. Mogelijke zoektermen zijn onder andere `speech to text', `speech to text using deep learning', `voice synthesis',\dots. Door deze zoektermen te gebruiken, zullen we een long list kunnen opstellen van mogelijke bronnen.
    \item Kritische evaluatie van bronnen: Hier zullen we de bronnen evalueren op basis van de titel, abstract en keywords. Zo zullen we een beter beeld krijgen van de bronnen en kunnen we bepalen of onze zoektermen relevant zijn voor het onderzoek. We zullen deze evalueren op basis van de volgende criteria: relevantie, actualiteit, biedt deze inzicht in het onderzoek, is het een betrouwbare bron, is het een peer-reviewed artikel, is het een wetenschappelijk artikel en is het afkomstig van een gerenommeerde instelling.
    \item Short list: De short list zal bestaan uit de bronnen die het meest aansluiten en relevant zijn voor het onderzoek.
    \item Synthese van bronnen: In deze stap zullen we de bronnen samenvatten in een doorlopende tekst. Die zal dienen als basis voor de literatuurstudie.
    \item Conclusie: De conclusie zal bestaan uit een samenvatting van de literatuurstudie op basis van de synthese van de bronnen. Hierin zal de onderzoeksvraag beantwoord worden.
    \item Scriptie: De scriptie zal bestaan uit de volgende onderdelen: inleiding, literatuurstudie, methodologie, resultaten, conclusie en referenties.
\end{itemize}

\par Daarna wordt de methodologie vervolgd met de identificatie van commerciële speech to text systemen en de bepaling van een baseline voor prestatievergelijking. De tests om de baseline vast te stellen omvatten het transcriberen van gesprekken in zowel standaard als niet-standaard Nederlands, gevolgd door de berekening van metrieken zoals Word Error Rate en Character Error Rate met Python en het jiwer package.
Deze stap zal volgende onderdelen bevatten:
\begin{itemize}
    \item Identificeer bestaande commerciële speech to text systemen die Nederlands ondersteunen.
    \item Bepaal de kostprijs en licentievoorwaarden van deze systemen.
    \item Evalueer de prestaties van deze systemen door het transcriberen van gesprekken in niet-standaard Nederland
\end{itemize}   
\par Op basis van de eerder gevonden systemen zullen we nu opzoek gaan naar vrij beschikbare alternatieven/open source systemen.
\begin{itemize}
    \item Onderzoek de beschikbaarheid van open source spraak naar tekstsystemen voor Nederlands.
    \item Analyseer de technische vereisten en mogelijke beperkingen van deze systemen.
    \item Voer tests uit om de prestaties van deze systemen voor de transcribering van niet-standaard Nederlands te beoordelen.
\end{itemize}
\par Steunend op deze gevonden systemen/modellen kunnen we nu gaan kijken of we ze kunnen finetunen voor niet-standaard Nederlands. Deze stap zal vooral plaatsvinden in python waar we gaan kijken of er de mogelijkheid is om de systemen te finetunen voor specifieke dialecten en informeel Nederlands. Dit kan zijn door bijv.: een hidden laag toevoegen, hyperparamter tuning, transfer learning, ... Om dan uiteindelijk het effect van de uitgevoerde finetuning te gaan testen op de prestaties en nauwkeurigheid.

\par Indien we een systeem vinden dat aan de vereisten voldoet kunnen we gebruikmaken van annotaties voor sprekersidentificatie. Hierbij onderzoeken we eerst de mogelijkheid van annotaties om de sprekers in het gesprek te identificeren. Indien we een mogelijk model gevonden hebben voor deze taak kunnen we het implementeren en de prestaties van de systemen met en zonder annotaties vergelijken.

\par Tenslotte zullen we kijken of er verbetering mogelijk is met bijkomende software (LLM) hierbij onderzoeken we de bruikbaarheid van Language Models (LLMs) of andere bijkomende software om automatisch gegenereerde teksten te verbeteren voor niet-standaard Nederlands.


\par De methodologie voor dit onderzoek omvat een uitgebreide aanpak, bestaande uit het vinden van relevante literatuur, het identificeren en evalueren van commerciële en open source speech to text systemen, het finetunen van systemen, sprekersidentificatie met annotaties en de mogelijke verbetering van resultaten met Language Models (LLMs). Deze methodologie is gebaseerd op experimenten en vergelijkende studies om de prestaties en nauwkeurigheid van speech to text systemen te verbeteren, vooral in niet-standaard Nederlands.

%---------- Verwachte resultaten ----------------------------------------------
\section{Verwacht resultaat, conclusie}%
\label{sec:verwachte_resultaten}

De initiële hypothese luidt dat bestaande spraak naar tekstsystemen, oorspronkelijk getraind op standaard Nederlands, moeite zullen ondervinden bij het nauwkeurig omzetten van niet-standaard varianten.
Uit de bevindingen in dit rapport blijkt overtuigend dat er aanzienlijke ruimte is voor verbetering van spraak-naar-tekstsystemen. De verwachting is dat een transfer learning model, ondersteund door Language Models (LLM's), aanzienlijke verbeteringen zal laten zien bij het omzetten van niet-standaard Nederlands. Deze verwachting wordt ondersteund door de reeds benadrukte effectiviteit van transfer learning op kleinere datasets, evenals het groeiende succes van LLM's die bij vorige onderzoeken nog niet beschikbaar waren.

Het is belangrijk op te merken dat indien de resultaten afwijken van de hierboven vermelde hypothese, dit nog altijd waardevolle inzichten oplevert. In dat geval kunnen we een vergelijking maken tussen de verwachte en daadwerkelijke resultaten, wat een boeiende kans biedt om te onderzoeken waarom het met de huidige technologie nog altijd uitdagend blijkt te zijn om niet-standaard Nederlands om te zetten.



%%---------- Andere bijlagen --------------------------------------------------
% TODO: Voeg hier eventuele andere bijlagen toe. Bv. als je deze BP voor de
% tweede keer indient, een overzicht van de verbeteringen t.o.v. het origineel.
%\input{...}

%%---------- Backmatter, referentielijst ---------------------------------------

\backmatter{}

\setlength\bibitemsep{2pt} %% Add Some space between the bibliograpy entries
\printbibliography[heading=bibintoc]

\end{document}
