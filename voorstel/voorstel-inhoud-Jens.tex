%---------- Inleiding ---------------------------------------------------------
\section{Introductie}%
\label{sec:introductie}
\epigraph{Nieuwe spraakherkenningstechnologieën en -methodes zullen een centraal onderdeel worden van het toekomstige leven omdat ze veel communicatietijd besparen en het gewone sms'en vervangen door spraak/audio.}{Natalya Shakhovska}

In het bovenstaande voorbeeld wordt duidelijk geïllustreerd hoe belangrijk speech-to-text is in ons dagelijks leven, zowel in het nu als in de toekomst. Deze technologie speelt een steeds grotere rol in onze snel evoluerende wereld doordat het ons in staat stelt gesproken taal nauwkeurig om te zetten in geschreven tekst. De invloed van speech-to-text is tegenwoordig al in gebruik met een breed scala aan toepassingen zoals:
\begin{itemize}
    \item kinderen helpen met problemen in gesproken, geschreven communicatie \cite{Kambouri2023}.
    \item automatische ondertiteling
    \item Google Assistant, Siri.
\end{itemize}

In dit onderzoek ligt de focus voornamelijk op spraakopdrachten, zoals die in gesprekken tussen klant en bedrijf of tussen vrienden voorkomen. Hoewel dergelijke systemen al aanzienlijke vooruitgang hebben geboekt, blijft de verwerking van dialecten en regionale accenten een uitdaging voor deze technologieën.

\subsection{Uitdagingen bij Niet-standaard Nederlands}
Het Nederlands taalgebied heeft een rijke variatie aan dialecten. Dat vormt dus een uitdaging voor speech to text systemen die oorspronkelijk getraind zijn op standaard Nederlands. De vraag die dan gesteld kan worden is: Is deze technologie instaat om deze variatie en diversiteit te begrijpen, en of ze kunnen worden verbeterd om nauwkeurige transcripties te leveren voor niet-standaard Nederlands?
Deze vraag zal de kern vormen van ons onderzoek en zal dienen als leidraad

Vanuit bovenstaand idee kwam het doel om een effectieve oplossing te bieden voor de problematische transcriptie van spraak in niet-standaard Nederlands, waarbij de focus ligt op dialecten, regionale accenten en het identificeren van de spreker, ook wanneer het resultaat niet is zo als we verwacht hadden kunnen we toch inzichten verwerven en een nuttige bijdrage leveren.

Het concrete resultaat zal bestaan uit een (aangepast) model voor spraakherkenning dat geoptimaliseerd is voor niet-standaard Nederlands. Echter, in het geval dat het ontwikkelen of aanpassen van een dergelijk model niet succesvol blijkt te zijn, zal het onderzoeksrapport een gedetailleerde evaluatie bevatten van de uitdagingen en beperkingen die zijn geïdentificeerd.

\subsection{Opzet van het onderzoek}
Dit onderzoek zal zich ontvouwen in verschillende onderdelen, beginnend met een grondige literatuurstudie om de huidige stand van zaken in het vakgebied te begrijpen. Vervolgens zal de methodologie worden beschreven, waarbij we onze aanpak voor het verkennen en verbeteren van de transcriptiekwaliteit zullen uitschrijven. Ten slotte zullen we de verwachte resultaten bespreken.



%---------- Stand van zaken ---------------------------------------------------

\section{State-of-the-art}%
\label{sec:state-of-the-art}

Speech to text technologie ook gekend als speech recognition, heeft al een opvallende vooruitgang geboekt over verschillende domeinen heen. Dit kunnen we bevestigen uit het onderzoek van \cite{Kambouri2023} die aantoont hoe kinderen met schrijfproblemen voordeel kunnen halen uit deze technologie. Het verbeterde de kwaliteit en hoeveelheid van hun geschreven en gesproken communicatie.

Niettemin zien we uit het onderzoek van \cite{ajami2016use} dat deze technologie zeer succesvol is in de medische sector, waar het veel tijd en geld kan besparen. Deze sector heeft vaak te kampen met onjuiste registratie van gegevens, onleesbaarheid en ongeldige gegevens. Deze beperkingen brengen de kwaliteit van medische dossiers in gevaar en kunnen ernstige gevolgen hebben. Het onderzoek benadrukt dat spraakherkenningstechnologie de tijd die nodig is om medische dossiers in te vullen aanzienlijk kan verkorten, wat leidt tot een hogere productiviteit en lagere documentatiekosten.

\subsection{Uitdagingen}
Een van de belangrijkste uitdagingen die uit onderzoeken naar voren komen is de moeilijkheid om regionale accenten en gemengde talen te herkennen. 
Zoals beschreven door \cite{Reddy2022}, vormt meertalige speech to text technologie een unieke uitdaging, vooral als mensen zinnen gebruiken die een mix van talen bevatten. 
Dit vormt een obstakel voor effectieve communicatie tussen sprekers van verschillende talen of dialecten. Daarnaast wordt ook in het onderzoek van \cite{ajami2016use} benadrukt dat omgevingsgeluid problemen kan veroorzaken met de nauwkeurigheid van spraakherkenning, waardoor verdere verbeteringen op dit gebied nodig zijn.
Deze uitdagingen geven aanleiding tot belangrijke vragen voor toekomstig onderzoek: Kunnen we speech to text technologie verbeteren om beter om te gaan met regionale accenten en dialecten? Is er een manier om omgevingsgeluid te verminderen om de nauwkeurigheid van de technologie te vergroten?


\subsection{Vooruitgang}
Spraak is 1 van de meest gebruikte vormen van communicatie en heeft daarom ook een groot aantal toepassingen, zeker voor analfabeten en zorginstellingen \cite{Arun2021}. Volgens het onderzoek van \cite{Roepke2019} kan het gebruik van speech to text technologie een grote invloed hebben op de maatschappij. Datzelfde onderzoek heeft ook het belang van transfer learning aangeduid voor kleinere datasets te gebruiken en toch een succesvol resultaat te behalen. Uit zowat alle onderzoeken wordt de behoefte aan dergelijke systemen benadrukt, en onderstreept de rol van spraak- en taalverwerking om computers instaat te stellen menselijke talen te begrijpen en te interpreteren.

\subsection{Conclusie}
Speech to text technologie heeft het potentieel om vooruitgang te boeken in diverse sectoren, waaronder gezondheidszorg, onderwijs en interculturele communicatie. Zoals deze literatuurstudie heeft aangetoond, zijn er toch nog aanzienlijke uitdagingen die moeten worden overwonnen. Toekomstig onderzoek kan gericht zijn op het verbeteren van de nauwkeurigheid van de technologie, het overbruggen van taalbarrières en het verbeteren van communicatie voor mensen met speciale behoeften.

Deze literatuurstudie sluit nauw aan bij het gekozen onderzoek, dat zich richt op het optimaliseren van transcriptie voor niet-standaard Nederlands in gesprekken. Daarnaast biedt het de mogelijkheid om te onderzoeken hoe omgevingsgeluid effectief kan worden gedempt om de nauwkeurigheid van speech to text technologie te vergroten.


%---------- Methodologie ------------------------------------------------------
\section{Methodologie}%
\label{sec:methodologie}

We zullen het onderzoek uitvoeren in zes stappen, waarbij de eerste fase gericht is op het vinden van relevante bronnen voor het onderzoek. De volgende stappen worden dan gevolgd.

\begin{itemize}
    \item Long list: De long list zal bestaan uit alle bronnen die mogelijk relevant zijn voor het onderzoek. Mogelijke zoektermen zijn onder andere `speech to text', `speech to text using deep learning', `voice synthesis',\dots. Door deze zoektermen te gebruiken, zullen we een long list kunnen opstellen van mogelijke bronnen.
    \item Kritische evaluatie van bronnen: Hier zullen we de bronnen evalueren op basis van de titel, abstract en keywords. Zo zullen we een beter beeld krijgen van de bronnen en kunnen we bepalen of onze zoektermen relevant zijn voor het onderzoek. We zullen deze evalueren op basis van de volgende criteria: relevantie, actualiteit, biedt deze inzicht in het onderzoek, is het een betrouwbare bron, is het een peer-reviewed artikel, is het een wetenschappelijk artikel en is het afkomstig van een gerenommeerde instelling.
    \item Short list: De short list zal bestaan uit de bronnen die het meest aansluiten en relevant zijn voor het onderzoek.
    \item Synthese van bronnen: In deze stap zullen we de bronnen samenvatten in een doorlopende tekst. Die zal dienen als basis voor de literatuurstudie.
    \item Conclusie: De conclusie zal bestaan uit een samenvatting van de literatuurstudie op basis van de synthese van de bronnen. Hierin zal de onderzoeksvraag beantwoord worden.
    \item Scriptie: De scriptie zal bestaan uit de volgende onderdelen: inleiding, literatuurstudie, methodologie, resultaten, conclusie en referenties.
\end{itemize}

\par Daarna wordt de methodologie vervolgd met de identificatie van commerciële speech to text systemen en de bepaling van een baseline voor prestatievergelijking. De tests om de baseline vast te stellen omvatten het transcriberen van gesprekken in zowel standaard als niet-standaard Nederlands, gevolgd door de berekening van metrieken zoals Word Error Rate en Character Error Rate met Python en het jiwer package.
Deze stap zal volgende onderdelen bevatten:
\begin{itemize}
    \item Identificeer bestaande commerciële speech to text systemen die Nederlands ondersteunen.
    \item Bepaal de kostprijs en licentievoorwaarden van deze systemen.
    \item Evalueer de prestaties van deze systemen door het transcriberen van gesprekken in niet-standaard Nederland
\end{itemize}   
\par Op basis van de eerder gevonden systemen zullen we nu opzoek gaan naar vrij beschikbare alternatieven/open source systemen.
\begin{itemize}
    \item Onderzoek de beschikbaarheid van open source spraak naar tekstsystemen voor Nederlands.
    \item Analyseer de technische vereisten en mogelijke beperkingen van deze systemen.
    \item Voer tests uit om de prestaties van deze systemen voor de transcribering van niet-standaard Nederlands te beoordelen.
\end{itemize}
\par Steunend op deze gevonden systemen/modellen kunnen we nu gaan kijken of we ze kunnen finetunen voor niet-standaard Nederlands. Deze stap zal vooral plaatsvinden in python waar we gaan kijken of er de mogelijkheid is om de systemen te finetunen voor specifieke dialecten en informeel Nederlands. Dit kan zijn door bijv.: een hidden laag toevoegen, hyperparamter tuning, transfer learning, ... Om dan uiteindelijk het effect van de uitgevoerde finetuning te gaan testen op de prestaties en nauwkeurigheid.

\par Indien we een systeem vinden dat aan de vereisten voldoet kunnen we gebruikmaken van annotaties voor sprekersidentificatie. Hierbij onderzoeken we eerst de mogelijkheid van annotaties om de sprekers in het gesprek te identificeren. Indien we een mogelijk model gevonden hebben voor deze taak kunnen we het implementeren en de prestaties van de systemen met en zonder annotaties vergelijken.

\par Tenslotte zullen we kijken of er verbetering mogelijk is met bijkomende software (LLM) hierbij onderzoeken we de bruikbaarheid van Language Models (LLMs) of andere bijkomende software om automatisch gegenereerde teksten te verbeteren voor niet-standaard Nederlands.


\par De methodologie voor dit onderzoek omvat een uitgebreide aanpak, bestaande uit het vinden van relevante literatuur, het identificeren en evalueren van commerciële en open source speech to text systemen, het finetunen van systemen, sprekersidentificatie met annotaties en de mogelijke verbetering van resultaten met Language Models (LLMs). Deze methodologie is gebaseerd op experimenten en vergelijkende studies om de prestaties en nauwkeurigheid van speech to text systemen te verbeteren, vooral in niet-standaard Nederlands.

%---------- Verwachte resultaten ----------------------------------------------
\section{Verwacht resultaat, conclusie}%
\label{sec:verwachte_resultaten}

De initiële hypothese luidt dat bestaande spraak naar tekstsystemen, oorspronkelijk getraind op standaard Nederlands, moeite zullen ondervinden bij het nauwkeurig omzetten van niet-standaard varianten.
Uit de bevindingen in dit rapport blijkt overtuigend dat er aanzienlijke ruimte is voor verbetering van spraak-naar-tekstsystemen. De verwachting is dat een transfer learning model, ondersteund door Language Models (LLM's), aanzienlijke verbeteringen zal laten zien bij het omzetten van niet-standaard Nederlands. Deze verwachting wordt ondersteund door de reeds benadrukte effectiviteit van transfer learning op kleinere datasets, evenals het groeiende succes van LLM's die bij vorige onderzoeken nog niet beschikbaar waren.

Het is belangrijk op te merken dat indien de resultaten afwijken van de hierboven vermelde hypothese, dit nog altijd waardevolle inzichten oplevert. In dat geval kunnen we een vergelijking maken tussen de verwachte en daadwerkelijke resultaten, wat een boeiende kans biedt om te onderzoeken waarom het met de huidige technologie nog altijd uitdagend blijkt te zijn om niet-standaard Nederlands om te zetten.

