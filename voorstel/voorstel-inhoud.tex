%---------- Inleiding ---------------------------------------------------------


\section{Introductie}%
\label{sec:introductie}
\epigraph{Nieuwe spraakherkenningstechnologieën en -methodes zullen een centraal onderdeel worden van 
    het toekomstige leven omdat ze veel communicatietijd besparen en het gewone sms'en 
    vervangen door spraak/audio.}{Natalya Shakhovska}
    
De tekst hierboven toont dus hoe belangrijk speech-to-text kan zijn. 
Speech-to-text speelt een steeds prominentere rol in deze snel evoluerende tijd, het stelt ons in staat om gesproken taal nauwkeurig om te zetten in geschreven tekst. De technologie heeft een breed scala van toepassingen zoals:
\begin{itemize}
    \item kinderen helpen met problemen in gesproken en geschreven communicatie.\autocite{Kambouri2023}
    \item automatische ondertiteling
    \item ...
\end{itemize}

Hier zullen we ons vooral focussen op spraakopdrachten zoals een gesprek tussen klant en bedrijf, 2 vrienden etc.
Ondanks dat dergelijke systemen al een grote vooruitgang geboekt hebben blijven ze de verwerking van dialecten, regionale accenten een uitdaging vinden.

\subsection{Uitdagingen bij Niet-standaard Nederlands}
Het Nederlands taalgebied heeft een rijke variatie aan dialecten. Dat vormt dus een uitdaging voor speech-to-text systemen die oorspronkelijk getraind zijn op standaard Nederlands. De vraag die dan gesteld kan worden is: Is deze technologie in staat om deze variatie en diversiteit te begrijpen, en of ze kunnen worden geoptimaliseerd om nauwkeurige transcripties te leveren voor niet-standaard Nederlands ?
Deze vraag zal de kern vormen van ons onderzoek en zal dienen als leidraad.

Het doel is dus om een effectieve oplossing te bieden voor de problematische transcriptie van spraakgegevens in niet-standaard Nederlands, waarbij de focus ligt op dialecten en regionale accenten, ook indien het resultaat niet is zo als verwacht kunnen we toch inzichten geven waarom dit net zo moeilijk is. 

Het concrete resultaat zal bestaan uit een werkend prototype of een aangepast model voor spraakherkenning dat geoptimaliseerd is voor niet-standaard Nederlands. Echter, in het geval dat de ontwikkeling of aanpassing van een dergelijk model niet succesvol blijkt te zijn, zal het onderzoeksrapport een gedetailleerde evaluatie bevatten van de uitdagingen en beperkingen die zijn geïdentificeerd.

\subsection{Opzet van het onderzoek}
Dit onderzoek zal zich ontvouwen in verschillende secties, beginnend met een grondige literatuurstudie om de huidige stand van zaken in het vakgebied te begrijpen. Vervolgens zal de methodologie worden beschreven, waarbij we onze aanpak voor het verkennen en verbeteren van de transcriptiekwaliteit zullen uiteenzetten. Ten slotte zullen we de verwachte resultaten bespreken.



%---------- Stand van zaken ---------------------------------------------------

\section{State-of-the-art}%
\label{sec:state-of-the-art}

Hier beschrijf je de \emph{state-of-the-art} rondom je gekozen onderzoeksdomein, d.w.z.\ een inleidende, doorlopende tekst over het onderzoeksdomein van je bachelorproef. Je steunt daarbij heel sterk op de professionele \emph{vakliteratuur}, en niet zozeer op populariserende teksten voor een breed publiek. Wat is de huidige stand van zaken in dit domein, en wat zijn nog eventuele open vragen (die misschien de aanleiding waren tot je onderzoeksvraag!)?

Je mag de titel van deze sectie ook aanpassen (literatuurstudie, stand van zaken, enz.). Zijn er al gelijkaardige onderzoeken gevoerd? Wat concluderen ze? Wat is het verschil met jouw onderzoek?

Verwijs bij elke introductie van een term of bewering over het domein naar de vakliteratuur, bijvoorbeeld! Denk zeker goed na welke werken je refereert en waarom.

Draag zorg voor correcte literatuurverwijzingen! Een bronvermelding hoort thuis \emph{binnen} de zin waar je je op die bron baseert, dus niet er buiten! Maak meteen een verwijzing als je gebruik maakt van een bron. Doe dit dus \emph{niet} aan het einde van een lange paragraaf. Baseer nooit teveel aansluitende tekst op eenzelfde bron.

Als je informatie over bronnen verzamelt in JabRef, zorg er dan voor dat alle nodige info aanwezig is om de bron terug te vinden (zoals uitvoerig besproken in de lessen Research Methods).

% Voor literatuurverwijzingen zijn er twee belangrijke commando's:
% \autocite{KEY} => (Auteur, jaartal) Gebruik dit als de naam van de auteur
%   geen onderdeel is van de zin.
% \textcite{KEY} => Auteur (jaartal)  Gebruik dit als de auteursnaam wel een
%   functie heeft in de zin (bv. ``Uit onderzoek door Doll & Hill (1954) bleek
%   ...'')

Je mag deze sectie nog verder onderverdelen in subsecties als dit de structuur van de tekst kan verduidelijken.

%---------- Methodologie ------------------------------------------------------
\section{Methodologie}%
\label{sec:methodologie}

Hier beschrijf je hoe je van plan bent het onderzoek te voeren. Welke onderzoekstechniek ga je toepassen om elk van je onderzoeksvragen te beantwoorden? Gebruik je hiervoor literatuurstudie, interviews met belanghebbenden (bv.~voor requirements-analyse), experimenten, simulaties, vergelijkende studie, risico-analyse, PoC, \ldots?

Valt je onderwerp onder één van de typische soorten bachelorproeven die besproken zijn in de lessen Research Methods (bv.\ vergelijkende studie of risico-analyse)? Zorg er dan ook voor dat we duidelijk de verschillende stappen terug vinden die we verwachten in dit soort onderzoek!

Vermijd onderzoekstechnieken die geen objectieve, meetbare resultaten kunnen opleveren. Enquêtes, bijvoorbeeld, zijn voor een bachelorproef informatica meestal \textbf{niet geschikt}. De antwoorden zijn eerder meningen dan feiten en in de praktijk blijkt het ook bijzonder moeilijk om voldoende respondenten te vinden. Studenten die een enquête willen voeren, hebben meestal ook geen goede definitie van de populatie, waardoor ook niet kan aangetoond worden dat eventuele resultaten representatief zijn.

Uit dit onderdeel moet duidelijk naar voor komen dat je bachelorproef ook technisch voldoen\-de diepgang zal bevatten. Het zou niet kloppen als een bachelorproef informatica ook door bv.\ een student marketing zou kunnen uitgevoerd worden.

Je beschrijft ook al welke tools (hardware, software, diensten, \ldots) je denkt hiervoor te gebruiken of te ontwikkelen.

Probeer ook een tijdschatting te maken. Hoe lang zal je met elke fase van je onderzoek bezig zijn en wat zijn de concrete \emph{deliverables} in elke fase?

%---------- Verwachte resultaten ----------------------------------------------
\section{Verwacht resultaat, conclusie}%
\label{sec:verwachte_resultaten}

Hier beschrijf je welke resultaten je verwacht. Als je metingen en simulaties uitvoert, kan je hier al mock-ups maken van de grafieken samen met de verwachte conclusies. Benoem zeker al je assen en de onderdelen van de grafiek die je gaat gebruiken. Dit zorgt ervoor dat je concreet weet welk soort data je moet verzamelen en hoe je die moet meten.

Wat heeft de doelgroep van je onderzoek aan het resultaat? Op welke manier zorgt jouw bachelorproef voor een meerwaarde?

Hier beschrijf je wat je verwacht uit je onderzoek, met de motivatie waarom. Het is \textbf{niet} erg indien uit je onderzoek andere resultaten en conclusies vloeien dan dat je hier beschrijft: het is dan juist interessant om te onderzoeken waarom jouw hypothesen niet overeenkomen met de resultaten.

