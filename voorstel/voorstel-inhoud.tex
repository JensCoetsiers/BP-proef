%---------- Inleiding ---------------------------------------------------------

\section{Introductie}%
\label{sec:introductie}
\epigraph{Nieuwe spraakherkenningstechnologieën en -methodes zullen een centraal onderdeel worden van 
    het toekomstige leven omdat ze veel communicatietijd besparen en het gewone sms'en 
    vervangen door spraak/audio.}{Natalya Shakhovska}
    
In het bovenstaande voorbeeld wordt duidelijk geïllustreerd hoe belangrijk speech to text is in ons dagelijks leven, zowel in het nu als in de toekomst. Deze technologie speelt een steeds grotere rol in onze snel evoluerende wereld doordat het ons instaat stelt gesproken taal nauwkeurig om te zetten in geschreven tekst. De invloed van speech to text is tegenwoordig al in gebruik met een breed scala aan toepassingen zoals:
\begin{itemize}
    \item kinderen helpen met problemen in gesproken en geschreven communicatie.\autocite{Kambouri2023}
    \item automatische ondertiteling
    \item Google assistant, Siri.
    \item ...
\end{itemize}

In dit onderzoek ligt de focus voornamelijk op spraakopdrachten, zoals die in gesprekken tussen klant en bedrijf of tussen vrienden voorkomen. Hoewel dergelijke systemen al aanzienlijke vooruitgang hebben geboekt, blijft de verwerking van dialecten en regionale accenten een uitdaging voor deze technologieën.

\subsection{Uitdagingen bij Niet-standaard Nederlands}
Het Nederlands taalgebied heeft een rijke variatie aan dialecten. Dat vormt dus een uitdaging voor speech to text systemen die oorspronkelijk getraind zijn op standaard Nederlands. De vraag die dan gesteld kan worden is: Is deze technologie instaat om deze variatie en diversiteit te begrijpen, en of ze kunnen worden verbeterd om nauwkeurige transcripties te leveren voor niet-standaard Nederlands?
Deze vraag zal de kern vormen van ons onderzoek en zal dienen als leidraad

Vanuit bovenstaand idee kwam het doel om een effectieve oplossing te bieden voor de problematische transcriptie van spraak in niet-standaard Nederlands, waarbij de focus ligt op dialecten, regionale accenten en het identificeren van de spreker, ook wanneer het resultaat niet is zo als we verwacht hadden kunnen we toch inzichten verwerven en een nuttige bijdrage leveren.

Het concrete resultaat zal bestaan uit een (aangepast) model voor spraakherkenning dat geoptimaliseerd is voor niet-standaard Nederlands. Echter, in het geval dat het ontwikkelen of aanpassen van een dergelijk model niet succesvol blijkt te zijn, zal het onderzoeksrapport een gedetailleerde evaluatie bevatten van de uitdagingen en beperkingen die zijn geïdentificeerd.

\subsection{Opzet van het onderzoek}
Dit onderzoek zal zich ontvouwen in verschillende onderdelen, beginnend met een grondige literatuurstudie om de huidige stand van zaken in het vakgebied te begrijpen. Vervolgens zal de methodologie worden beschreven, waarbij we onze aanpak voor het verkennen en verbeteren van de transcriptiekwaliteit zullen uitschrijven. Ten slotte zullen we de verwachte resultaten bespreken.



%---------- Stand van zaken ---------------------------------------------------

\section{State-of-the-art}%
\label{sec:state-of-the-art}

Hier beschrijf je de \emph{state-of-the-art} rondom je gekozen onderzoeksdomein, d.w.z.\ een inleidende, doorlopende tekst over het onderzoeksdomein van je bachelorproef. Je steunt daarbij heel sterk op de professionele \emph{vakliteratuur}, en niet zozeer op populariserende teksten voor een breed publiek. Wat is de huidige stand van zaken in dit domein, en wat zijn nog eventuele open vragen (die misschien de aanleiding waren tot je onderzoeksvraag!)?

Je mag de titel van deze sectie ook aanpassen (literatuurstudie, stand van zaken, enz.). Zijn er al gelijkaardige onderzoeken gevoerd? Wat concluderen ze? Wat is het verschil met jouw onderzoek?

Verwijs bij elke introductie van een term of bewering over het domein naar de vakliteratuur, bijvoorbeeld! Denk zeker goed na welke werken je refereert en waarom.

Draag zorg voor correcte literatuurverwijzingen! Een bronvermelding hoort thuis \emph{binnen} de zin waar je je op die bron baseert, dus niet er buiten! Maak meteen een verwijzing als je gebruik maakt van een bron. Doe dit dus \emph{niet} aan het einde van een lange paragraaf. Baseer nooit teveel aansluitende tekst op eenzelfde bron.

Als je informatie over bronnen verzamelt in JabRef, zorg er dan voor dat alle nodige info aanwezig is om de bron terug te vinden (zoals uitvoerig besproken in de lessen Research Methods).

% Voor literatuurverwijzingen zijn er twee belangrijke commando's:
% \autocite{KEY} => (Auteur, jaartal) Gebruik dit als de naam van de auteur
%   geen onderdeel is van de zin.
% \textcite{KEY} => Auteur (jaartal)  Gebruik dit als de auteursnaam wel een
%   functie heeft in de zin (bv. ``Uit onderzoek door Doll & Hill (1954) bleek
%   ...'')

Je mag deze sectie nog verder onderverdelen in subsecties als dit de structuur van de tekst kan verduidelijken.

%---------- Methodologie ------------------------------------------------------
\section{Methodologie}%
\label{sec:methodologie}

We zullen het onderzoek uitvoeren in zes stappen, waarbij de eerste fase gericht is op het vinden van relevante bronnen voor het onderzoek. De volgende stappen worden dan gevolgd.
\begin{itemize}
    \item Long list: De long list zal bestaan uit alle bronnen die mogelijk relevant zijn voor het onderzoek. Mogelijke zoektermen zijn onder andere 'speech to text', 'speech to text using deep learning', 'voice synthesis', ... . Door deze zoektermen te gebruiken, zullen we een long list kunnen opstellen van mogelijke bronnen.
    \item Kritische evaluatie van bronnen: Hier zullen we de bronnen evalueren op basis van de titel, abstract en keywords. Zo zullen we een beter beeld krijgen van de bronnen en kunnen we bepalen of onze zoektermen relevant zijn voor het onderzoek. We zullen deze evalueren op basis van de volgende criteria: relevantie, actualiteit, biedt deze inzicht in het onderzoek, is het een betrouwbare bron, is het een peer-reviewed artikel, is het een wetenschappelijk artikel en is het afkomstig van een gerenommeerde instelling.
    \item Short list: De short list zal bestaan uit de bronnen die het meest aansluiten en relevant zijn voor het onderzoek.
    \item Synthese van bronnen: In deze stap zullen we de bronnen samenvatten in een doorlopende tekst. Die zal dienen als basis voor de literatuurstudie.
    \item Conclusie: De conclusie zal bestaan uit een samenvatting van de literatuurstudie op basis van de synthese van de bronnen. Hierin zal de onderzoeksvraag beantwoord worden.
    \item Scriptie: De scriptie zal bestaan uit de volgende onderdelen: inleiding, literatuurstudie, methodologie, resultaten, conclusie en referenties.
\end{itemize}

\par Daarna wordt de methodologie vervolgd met de identificatie van commerciële speech to text systemen en de bepaling van een baseline voor prestatievergelijking. De tests om de baseline vast te stellen omvatten het transcriberen van gesprekken in zowel standaard als niet-standaard Nederlands, gevolgd door de berekening van metrieken zoals Word Error Rate en Character Error Rate met Python en het jiwer package.
Deze stap zal volgende onderdelen bevatten:
\begin{itemize}
    \item Identificeer bestaande commerciële speech to text systemen die Nederlands ondersteunen.
    \item Bepaal de kostprijs en licentievoorwaarden van deze systemen.
    \item Evalueer de prestaties van deze systemen door het transcriberen van gesprekken in niet-standaard Nederland
\end{itemize}   
\par Op basis van de eerder gevonden systemen zullen we nu opzoek gaan naar vrij beschikbare alternatieven/open source systemen.
\begin{itemize}
    \item Onderzoek de beschikbaarheid van open source spraak naar tekstsystemen voor Nederlands.
    \item Analyseer de technische vereisten en mogelijke beperkingen van deze systemen.
    \item Voer tests uit om de prestaties van deze systemen voor de transcribering van niet-standaard Nederlands te beoordelen.
\end{itemize}
\par Steunend op deze gevonden systemen/modellen kunnen we nu gaan kijken of we ze kunnen finetunen voor niet-standaard Nederlands. Deze stap zal vooral plaatsvinden in python waar we gaan kijken of er de mogelijkheid is om de systemen te finetunen voor specifieke dialecten en informeel Nederlands. Dit kan zijn door bijv.: een hidden laag toevoegen, hyperparamter tuning, transfer learning, ... Om dan uiteindelijk het effect van de uitgevoerde finetuning te gaan testen op de prestaties en nauwkeurigheid.

\par Indien we een systeem vinden dat aan de vereisten voldoet kunnen we gebruikmaken van annotaties voor sprekersidentificatie. Hierbij onderzoeken we eerst de mogelijkheid van annotaties om de sprekers in het gesprek te identificeren. Indien we een mogelijk model gevonden hebben voor deze taak kunnen we het implementeren en de prestaties van de systemen met en zonder annotaties vergelijken.

\par Tenslotte zullen we kijken of er verbetering mogelijk is met bijkomende software (LLM) hierbij onderzoeken we de bruikbaarheid van Language Models (LLMs) of andere bijkomende software om automatisch gegenereerde teksten te verbeteren voor niet-standaard Nederlands.


\par De methodologie voor dit onderzoek omvat een uitgebreide aanpak, bestaande uit het vinden van relevante literatuur, het identificeren en evalueren van commerciële en open source speech to text systemen, het finetunen van systemen, sprekersidentificatie met annotaties en de mogelijke verbetering van resultaten met Language Models (LLMs). Deze methodologie is gebaseerd op experimenten en vergelijkende studies om de prestaties en nauwkeurigheid van speech to text systemen te verbeteren, vooral in niet-standaard Nederlands.

%---------- Verwachte resultaten ----------------------------------------------
\section{Verwacht resultaat, conclusie}%
\label{sec:verwachte_resultaten}

Hier beschrijf je welke resultaten je verwacht. Als je metingen en simulaties uitvoert, kan je hier al mock-ups maken van de grafieken samen met de verwachte conclusies. Benoem zeker al je assen en de onderdelen van de grafiek die je gaat gebruiken. Dit zorgt ervoor dat je concreet weet welk soort data je moet verzamelen en hoe je die moet meten.

Wat heeft de doelgroep van je onderzoek aan het resultaat? Op welke manier zorgt jouw bachelorproef voor een meerwaarde?

Hier beschrijf je wat je verwacht uit je onderzoek, met de motivatie waarom. Het is \textbf{niet} erg indien uit je onderzoek andere resultaten en conclusies vloeien dan dat je hier beschrijft: het is dan juist interessant om te onderzoeken waarom jouw hypothesen niet overeenkomen met de resultaten.

